\documentclass[11pt]{amsart}
\usepackage{amsmath, fancyhdr,multicol, etaremune, lastpage, xcolor, url,enumitem,graphics,titlesec, tabularx,longtable, booktabs}
%\usepackage{microtype}
\usepackage[hidelinks]{hyperref}
%\usepackage[left=.9in, right=.9in, top=.9in, bottom=.9in]{geometry}
\usepackage[margin=.9in]{geometry}

\usepackage[bitstream-charter]{mathdesign}
\usepackage{multicol}

\usepackage[utf8]{inputenc}
\usepackage[T1]{fontenc}

\definecolor{darkblue}{rgb}{0,0,.6}
\newcommand{\blue}{\color{darkblue}}
\setlength{\parindent}{0in}

\pagestyle{fancy}

\lhead{\blue\scshape Yufei Zhao}
%\leftmark
%\chead{\blue Yufei Zhao}
\rhead{\blue {\thepage}/\pageref{LastPage}}
%\rfoot{\tiny Updated: \today}
%\rightmark
%\renewcommand{\headrulewidth}{0pt}

\cfoot{}

%Section header formating
%\renewcommand{\section*}[1]{{\vspace{14pt}\blue \bfseries \large #1 \par \vspace{7pt}}}

\titleformat{\section}{\normalfont\bfseries\large\blue}{\thesection}{1em}{}
\titlespacing*{\section}{0em}{1em}{.2em}
\setlength{\parskip}{.4em}
\setlength{\extrarowheight}{.4em}
\setlength{\lineskip}{.2em}


% \newenvironment{newlist}{\begin{list}{$\bullet$} {\setlength{\leftmargin}{1.5em}}}{\end{list}}

% To add some paragraph space between lines.
% This also tells LaTeX to preferably break a page on one of these gaps
% if there is a needed pagebreak nearby.
%\newcommand{\blankline}{\quad\vspace{-6pt}\pagebreak[2]}


\newcommand{\rightloc}[1]{\hfill {\raggedright #1}}
\newcommand{\rightdate}[1]{\hfill {\raggedright #1}}
\renewcommand{\labelenumi}{\theenumi.}

\newcommand{\arXiv}[1]{\href{http://arxiv.org/abs/#1}{\color{black!50}\footnotesize\ttfamily  arXiv:#1}}

\newcommand{\p}[1]{{\bfseries #1}}
\renewcommand{\j}[1]{{\frenchspacing\itshape #1}}
\newcommand{\q}{\null\quad}

\begin{document}

\thispagestyle{empty}

\parbox{.5\textwidth}{\scalebox{3}{\sffamily\bfseries Yufei Zhao}}
\hfill
\parbox{.5\textwidth}
{  \begin{flushright} \small
\url{http://yufeizhao.com} \\
\url{yufeiz@mit.edu}
\\[2pt] \footnotesize
MIT Department of Mathematics \\
77 Massachusetts Ave, Room 2-271 \\
Cambridge, MA 02139, USA
\end{flushright}
}

\vspace{12pt}

{\blue
\hrule
\vspace{0.02in}
\hrule
\vspace{0.1in}
}

%\section*{Current Academic Position}

\p{Department of Mathematics, Massachusetts Institute of Technology} \rightloc{Cambridge, MA} \\
	\q Associate Professor 
		\rightdate{2022---\phantom{2022}} \\
	\q Assistant Professor
		\rightdate{2017---2022} \\
	\q Class of 1956 Career Development Assistant Professor
		\rightdate{2018---2021}
	
\section*{Previous and Visiting Academic Positions}

\p{Department of Mathematics, Stanford University} \rightloc{Stanford, CA} \\
	\q Visiting Assistant Professor \rightdate{Spring 2020}

\textbf{Simons Institute for the Theory of Computing, UC Berkeley} \rightloc{Berkeley, CA} \\
	\q Simons-Berkeley Research Fellow 
		\rightdate{Spring 2017}

\p{New College, University of Oxford} \rightloc{Oxford, UK} \\
	\q Esm\'ee Fairbairn Junior Research Fellow in Mathematics  
		\rightdate{2015---2017}

\section*{Education}

\p{Massachusetts Institute of Technology} \rightloc{Cambridge, MA}\\
\q Ph.D.~Mathematics. Advisor: Jacob Fox
\rightdate{2011---2015}

\p{University of Cambridge} \rightloc{Cambridge, UK} \\
\q M.A.St.~Mathematics with Distinction \rightdate{2010---2011}

\p{Massachusetts Institute of Technology} \rightloc{Cambridge, MA} \\
\q S.B.~Mathematics, with minor in Economics \rightdate{2006---2010} \\
\q S.B.~Computer Science and Engineering % GPA: $5.0/5.0$.



\section*{Selected Awards and Honors}

Edmund F. Kelly Research Award, MIT Mathematics, 2022

\textbf{NSF CAREER Award}, 2021

UROP Outstanding Mentor Award for Faculty, MIT, 2020

First Year Advisor Award---Innovative Seminar, MIT, 2019

\textbf{Sloan Research Fellowship}, 2019

Future of Science Award, MIT, 2018

\textbf{SIAM D\'enes K\"onig Prize}, 2018

Johnson Prize, MIT Mathematics, 2015

%\quad {\small in recognition of an outstanding mathematics paper accepted for publication in a major journal}

Microsoft Research PhD Fellowship, 2013--2015

% MIT Akamai Presidential Fellowship, 2011--2012

%Leslie Walshaw Prize, Examination Prize, and Senior Scholarship, Trinity College, Cambridge, 2011

% \quad {\small for top results in Cambridge Part III Mathematics examinations}

Morgan Prize Honorable Mention, 2011

%\quad {\small for outstanding research in mathematics by an undergraduate student}

Gates Cambridge Scholarship, 2010--2011

%\blankline

%Trinity College Cambridge Studentship in Mathematics, 2010

% \blankline

% Julie Payette---NSERC Research Scholarship (declined), 2010

Jon A.~Bucsela Prize in Mathematics, MIT Mathematics, 2010

%\quad {\small awarded to the top graduating senior in the MIT Mathematics Department}

%William Lowell Putnam Mathematics Competition

Putnam Math Competition: Three-time Putnam Fellow (top five rank) 2006, 2008, 2009; 7th Place 2007
%\item Member of MIT Team: First Place 2009, Third Place 2007 and 2008

% \blankline

% MIT 6.370 BattleCode AI Programming Competition: first place team 2008

International Mathematical Olympiad: Gold Medal 2005, Silver Medal
2006, Bronze Medal 2004

% \begin{newlist}
% \item Silver Medal, 2006  \rightloc{Ljubljana, Slovenia}
% \item Gold Medal, 2005    \rightloc{Merida, Mexico}
% \item Bronze Medal, 2004  \rightloc{Athens, Greece}
% \end{newlist}

%\blankline
%
%USA Mathematical Olympiad: Third Place 2005 and 2006
%
%\blankline
%
%Canadian Mathematical Olympiad: First Place 2004

%\blankline
%
%\textbf{Asian Pacific Mathematical Olympiad}
%
%\begin{newlist}
%\item Perfect score, 2006
%\end{newlist}

% \blankline
% Member of honor societies: Phi Beta Kappa, Eta Kappa Nu


\section*{Research Interests}

Combinatorics (extremal, probabilistic, additive, graph theory, discrete geometry)

\section*{Grants}

NSF CAREER award DMS-2044606 \rightdate{2021---2026}

Sloan Research Fellowship \rightdate{2019---2021}

MIT Solomon Buchsbaum Research Fund \rightdate{2018---\phantom{2020}}

NSF award DMS-1764176 \rightdate{2018---2021}

NSF award DMS-1362326 \rightdate{2017---2018}

\section*{Research Internships}

Microsoft Research New England \rightloc{Cambridge, MA} \\
\q Mentor: Henry Cohn \rightdate{Summers 2010, 2011, 2013, 2014}

Microsoft Research Theory Group
\rightloc{Redmond, WA} \\
\q Mentor: Eyal Lubetzky \rightdate{Summer 2012}

\section*{Book}

%\begin{itemize}[leftmargin=0.3in,itemsep=4pt,topsep=0pt,partopsep=0pt,parsep=0pt, resume]

\textit{Graph Theory and Additive Combinatorics}, 
under contract with Cambridge University Press
\\
{} \quad (intro graduate-level textbook; complete draft available online)

%\end{etaremune}

\section*{Papers}

\begin{etaremune}[leftmargin=0.3in,itemsep=4pt,topsep=0pt,partopsep=0pt,parsep=0pt]

\item
Dingding Dong, Nitya Mani, and Yufei Zhao \\
On the number of error correcting codes. \arXiv{2205.12363}


\item Yufei Zhao, Exploring a planet, revisited, \\
  \j{Amer. Math. Monthly}, to appear. \arXiv{2110.04376}

\item
Milan Haiman, Carl Schildkraut, Shengtong Zhang, and Yufei Zhao, \\
Graphs with high second eigenvalue multiplicity, \\
\j{Bull. Lond. Math. Soc.}, to appear. 
\arXiv{2109.13131}

\item 
Dingding Dong, Nitya Mani, and Yufei Zhao \\
Enumerating $k$-SAT functions,	\\
\j{ACM-SIAM Symposium on Discrete Algorithms (SODA)} 2022
\arXiv{2107.09233}

\item David Conlon, Jacob Fox, Benny Sudakov, and Yufei Zhao, \\
	Which graphs can be counted in $C_4$-free graphs? \arXiv{2106.03261}

\item Jacob Fox and Yufei Zhao, \\
	Removal lemmas and approximate homomorphisms,
	\\
	\j{Combin. Probab. Comput.}, to appear.
	\arXiv{2104.11626}

\item Aaron Berger and Yufei Zhao, \\
	$K_4$-intersecting families of graphs. 
	\arXiv{2103.12671}

\item Ashwin Sah, Mehtaab Sawhney, Yufei Zhao, \\
	  The cylindrical width of transitive sets, \\
	  \j{Israel J. Math.}, to appear. \arXiv{2101.11207}

\item Ashwin Sah, Mehtaab Sawhney, Yufei Zhao, \\
	  Paths of given length in tournaments. 
	\arXiv{2012.00262}

\item Jonathan Tidor, Hung-Hsun Hans Yu, and Yufei Zhao, \\
	  Joints of varieties, \\
	  \j{Geom. Funct. Anal.} 32 (2022) 302--339.
	\arXiv{2008.01610}

\item Matthew Kwan, Lisa Sauermann, and Yufei Zhao, \\
	  Extension complexity of low-dimensional polytopes, \\
	  \j{Trans. Amer. Math. Soc.} 375 (2022),  4209--4250. 
	\arXiv{2006.08836}

\item Zilin Jiang, Jonathan Tidor, Yuan Yao, Shengtong Zhang, and Yufei Zhao, \\
	  Spherical two-distance sets and eigenvalues of signed graphs, \\
	\j{Combinatorica}, to appear. 
	\arXiv{2006.06633}

\item Ashwin Sah, Mehtaab Sawhney, and Yufei Zhao, \\
	  Cayley graphs without a bounded eigenbasis, \\
	  \j{Int. Math. Res. Not. IMRN} 2022 (2022), 6157--6185. \arXiv{2005.04502}

\item Jacob Fox, Yuval Wigderson, and Yufei Zhao, \\
	  A short proof of the canonical polynomial van der Waerden theorem,\\
	  \j{C. R. Math. Acad. Sci. Paris} 358 (2020), 957--959.
	  \arXiv{2005.04135}

\item Jacob Fox, Huy Tuan Pham, and Yufei Zhao, \\
	  Tower-type bounds for Roth's theorem with popular differences.  \\
	\j{J. Eur. Math. Soc. (JEMS)}, to appear. 	
	\arXiv{2004.13690}

\item David Conlon, Jacob Fox, Benny Sudakov, and Yufei Zhao, \\
	The regularity method for graphs with few 4-cycles,  \\
	\j{J. Lond. Math. Soc.} 104 (2021), 2376--2401. \arXiv{2004.10180}

\item Ashwin Sah, Mehtaab Sawhney, and Yufei Zhao, \\
	Patterns without a popular difference, \\
	\j{Discrete Anal.}, 2021:8, 30 pp. \arXiv{2004.07722}

\item Ashwin Sah, Mehtaab Sawhney, Jonathan Tidor, and Yufei Zhao, \\
	 A counterexample to the Bollob\'as-Riordan conjectures on sparse graph limits, \\
	 \j{Combin. Probab. Comput.} 30 (2021), 796--799. \arXiv{2003.05272} 

\item Hung-Hsun Hans Yu and Yufei Zhao, \\
	Joints tightened, \\
	\j{Amer. J. Math.}, to appear. 
	\arXiv{1911.08605}

\item Jonathan Tidor and Yufei Zhao, \\
	Testing linear-invariant properties, \\
	\j{IEEE Symposium on Foundations of Computer Science (FOCS)} 2020, and \\
	\j{SIAM J. Comput.}, to appear. \arXiv{1911.06793}

\item Jacob Fox, Jonathan Tidor, and Yufei Zhao, \\
	Induced arithmetic removal: complexity 1 patterns over finite fields, \\
	\j{Israel J. Math.} 248 (2022), 1--38. \arXiv{1911.03427}

\item Jacob Fox, Huy Tuan Pham, and Yufei Zhao, \\
	Common and Sidorenko linear equations, \\
	\j{Q. J. Math.}  72 (2021), 1223--1234. \arXiv{1910.06436}

\item Yang Liu and Yufei Zhao, \\
	 On the upper tail problem for random hypergraphs, \\
	 \j{Random Structures Algorithms} 58 (2021), 179--220.
	  \arXiv{1910.02916}

\item Zilin Jiang, Jonathan Tidor, Yuan Yao, Shengtong Zhang, and Yufei Zhao, \\
	  Equiangular lines with a fixed angle, \\
      \j{Ann. of Math.} 194 (2021), 729--743. \arXiv{1907.12466}

\item Yufei Zhao and Yunkun Zhou, \\
	  Impartial digraphs, \\
	  \j{Combinatorica} 40 (2020), 875--896. \arXiv{1906.10482}

\item Ashwin Sah, Mehtaab Sawhney, David Stoner, and Yufei Zhao, \\
	Exponential improvements for superball packing upper bounds, \\
	\j{Adv.~Math.} 365 (2020), 107056.
	\arXiv{1904.11462}

\item Jacob Fox, Ashwin Sah, Mehtaab Sawhney, David Stoner, and Yufei Zhao, \\
	Triforce and corners, \\
	\j{Math.~Proc.~Cambridge Philos.~Soc.} 169 (2020), 209--223. \arXiv{1903.04863}

\item Ashwin Sah, Mehtaab Sawhney, David Stoner, and Yufei Zhao, \\
	A reverse Sidorenko inequality, \\
	\j{Invent. Math.} 221 (2020), 665--711.
	\arXiv{1809.09462}

\item David Conlon, Jonathan Tidor, and Yufei Zhao, \\
	Hypergraph expanders of all uniformities from Cayley graphs, \\
	\j{Proc. Lond. Math. Soc.} 121 (2020), 1311--1336.
	\arXiv{1809.06342}

\item Asaf Ferber, Vishesh Jain, and Yufei Zhao, \\
	On the number of Hadamard matrices via anti-concentration, \\
	\j{Combin. Probab. Comput.}, to appear.
	\arXiv{1808.07222}

\item Ashwin Sah, Mehtaab Sawhney, David Stoner, and Yufei Zhao, \\
The number of independent sets in an irregular graph, \\
	\j{J. Combin. Theory Ser. B} 138 (2019), 172--195. \arXiv{1805.04021}.

\item Jacob Fox, L\'aszl\'o Mikl\'os Lov\'asz, and Yufei Zhao, \\
  A fast new algorithm for weak graph regularity,  \\
  \j{Combin. Probab. Comput.} 28 (2019), 777--790.
  \arXiv{1801.05037}

\item Noga Alon, Jacob Fox, and Yufei Zhao, \\
Efficient arithmetic regularity and removal lemmas for induced bipartite patterns, \\
\j{Discrete Anal.} 2019:3, 14 pp. \arXiv{1801.04675}

\item Yufei Zhao, Group representations that resist worst-case sampling. \arXiv{1705.04675}

\item Yufei Zhao, Extremal regular graphs: independent sets and graph homomorphisms, \\
  \j{Amer. Math. Monthly} 124 (2017), 827--843.
  \arXiv{1610.09210}

\item Bhaswar B. Bhattacharya, Shirshendu Ganguly, Xuancheng Shao, and Yufei Zhao, \\
  Upper tails for arithmetic progressions in a random set, \\
  \j{Int. Math. Res. Not. IMRN} 2020, 167--213. 
  \arXiv{1605.02994}

\item Jacob Fox, L\'aszl\'o Mikl\'os Lov\'asz, and Yufei Zhao, \\
  On regularity lemmas and their algorithmic applications, \\
  \j{Combin. Probab. Comput.} 26 (2017), 481--505. 
  \arXiv{1604.00733}

\item David Conlon and Yufei Zhao, \\
  Quasirandom Cayley graphs, \\
  \j{Discrete Anal.} 2017:6, 14 pp.
  \arXiv{1603.03025}

\item Bhaswar B. Bhattacharya, Shirshendu Ganguly, Eyal Lubetzky, and Yufei Zhao, \\
  Upper tails and independence polynomials in random graphs,
  \\
  \j{Adv. Math.}  319 (2017), 313--347.
  \arXiv{1507.04074}

\item L\'aszl\'o Mikl\'os Lov\'asz and Yufei Zhao, \\
  On derivatives of graphon parameters, \\
  \j{J. Combin. Theory Ser. A} 145 (2017), 364--368.
  \arXiv{1505.07448}

\item Yufei Zhao,
  On the lower tail variational problem for random graphs, \\
  \j{Combin. Probab. Comput.} 26 (2017), 301--320.
  \arXiv{1502.00867}

\item Christian Borgs, Jennifer T. Chayes, Henry Cohn, and Yufei Zhao, \\
  An $L^p$ theory of sparse graph convergence II:
  LD convergence, quotients, and right convergence,
  \\
  \j{Ann. Probab.} 46 (2018), 337--396.
  \arXiv{1408.0744}

\item David Conlon, Jacob Fox, and Yufei Zhao, \\
  The Green-Tao theorem: an exposition, \\
  \j{EMS Surv.~Math.~Sci.} 1 (2014), 249--282.
  \arXiv{1403.2957}

\item Eyal Lubetzky and Yufei Zhao, \\
  On the variational problem for upper tails in sparse random graphs, \\
  \j{Random Structures Algorithms} 50 (2017), 420--436.
  \arXiv{1402.6011}

\item Christian Borgs, Jennifer T. Chayes, Henry Cohn, and Yufei Zhao, \\ 
  An $L^p$ theory of sparse graph convergence I: limits, sparse random
  graph models, and power law distributions, \\
  \j{Trans. Amer. Math. Soc.} 372 (2019), 3019--3062.
  \arXiv{1401.2906}

\item Yufei Zhao, 
  An arithmetic transference proof of a relative Szemer\'edi theorem, \\
  \j{Math.~Proc.~Cambridge Philos.~Soc.} 156 (2014), 255--261.
  \arXiv{1307.4959}

\item Jacob Fox and Yufei Zhao, \\
  A short proof of the multidimensional Szemer\'edi theorem in the primes, \\
  \j{Amer. J. Math.} 137 (2015), 1139--1145.
  \arXiv{1307.4679}

\item David Conlon, Jacob Fox, and Yufei Zhao, \\
  A relative Szemer\'edi theorem,  \\
  \j{Geom. Funct. Anal.} 25 (2015), 733--762.
  \arXiv{1305.5440}
  %With companion note: Linear forms from the Gowers uniformity norm,
  %\arXiv{1305.5565}

\item Yufei Zhao, 
  Hypergraph limits: a regularity approach, \\
  \j{Random Structures Algorithms} 47 (2015), 205--226. %Preprint
  \arXiv{1302.1634}

\item Henry Cohn and Yufei Zhao, \\
  Sphere packing bounds via spherical codes, \\
  \j{Duke Math.~J.} 163 (2014), 1965--2002.
  \arXiv{1212.5966}

\item Henry Cohn and Yufei Zhao, \\
  Universally optimal error-correcting codes, \\
  \j{IEEE Trans.~Inform.~Theory} 60 (2014), 7442--7450.
  \arXiv{1212.1913}

\item Eyal Lubetzky and Yufei Zhao, \\
  On replica symmetry of large deviations in random graphs, \\
  \j{Random Structures Algorithms} 47 (2015) 109--146.
  \arXiv{1210.7013}

\item Jacob Fox, Po-Shen Loh, and Yufei Zhao, \\
  The critical window for the classical Ramsey-Tur\'an problem, \\
  \j{Combinatorica} 35 (2015) 435--476.
  \arXiv{1208.3276}

\item David Conlon, Jacob Fox, and Yufei Zhao, \\
  Extremal results in sparse pseudorandom graphs, \\
  \j{Adv.~Math.} {256} (2014), 206--290.
  \arXiv{1204.6645}

\item Yufei Zhao, 
  {The bipartite swapping trick on graph homomorphisms}, \\
  \j{SIAM J.~Discrete Math.} 25 (2011), 660--680.
  \arXiv{1104.3704}
  
\item Yufei Zhao,
  {Sets characterized by the number of missing sums and differences}, \\
  \j{J.~Number Theory} 11 (2011), 2107--2134.
  \arXiv{0911.2292}

\item David Galvin and Yufei Zhao, \\
  {The number of independent sets in graphs with small maximum degree}, \\
  \j{Graphs Combin.} 27 (2011), 177--186.
  \arXiv{1007.4803}

\item Yufei Zhao,
  {Counting MSTD sets in finite abelian groups}, \\
  \j{J.~Number Theory} {130} (2010), 2308--2322.
  \arXiv{0911.2288}

\item Yufei Zhao,
  {Constructing numerical semigroups of a given genus},\\
  \j{Semigroup Forum} {80} (2010), 242--254.
  \arXiv{0910.2075}
  
\item Yufei Zhao,
  {Constructing MSTD sets using bidirectonal ballot sequences},\\
  \j{J.~Number Theory} {130} (2010), 1212--1220.
  \arXiv{0908.4442}

\item Yufei Zhao,
  {The number of independent sets in a regular graph}, \\
  \j{Combin.~Probab.~Comput.} {19} (2010), 315--320.
  \arXiv{0909.3354}

\item Yufei Zhao,
  {The coefficients of a truncated Fibonacci power series}, \\
  \j{Fibonacci Quart.} {46/47} (2009), 53--55.

\end{etaremune}

%\newpage 


\section*{Invited Talks}

\emph{Selected talk videos and slides can be found on my homepage: \url{https://yufeizhao.com}}

\begin{itemize}[leftmargin=.4in,itemsep=5pt,topsep=0pt,label={}]



\item[2022]
Conference on Random Structures \& Algorithms (RS\&A): \textbf{Plenary Speaker} \rightloc{Gniezno, Poland}
% 8/1-5/2022 Equiangular lines and eigenvalue multiplicities

\item 
International Congress of Chinese Mathematicians (ICCM): \textbf{Plenary Speaker}
\rightloc{Nanjing, China}
% 8/1-5/2022 Equiangular lines and eigenvalue multiplicities

\item 
Shandong University Mathematics seminar \rightloc{Online}
%6/14/2022 Progressions in sparse pseudorandom sets

\item
Workshop on Critical and Collective Effects in Graphs and Networks (CCEGN-V) \rightloc{Falmouth, MA}
%6/6/2022 Equiangular lines and eigenvalue multiplicities

\item Oberwolfach workshop: Combinatorics, Probability and Computing \rightloc{Oberwolfach, Germany}
%4/28/2022 Enumerating k-SAT functions

\item U. of Chicago Mathematics Colloquium
\rightloc{Chicago, IL}
%3/23/2022 Equiangular lines and eigenvalue multiplicities

\item U. of Chicago Combinatorics and Theoretical Computer
  Science Seminar
\rightloc{Chicago, IL}
%3/22/2022 Enumerating k-SAT functions

\item Johns Hopkins Applied Mathematics and Statistics Colloquium 
\rightloc{Online}
%2/17/2022 Equiangular lines and eigenvalue multiplicities

\item Vrije Universiteit in Amsterdam General Mathematics Colloquium \rightloc{Online}

%2/9/2022 Equiangular lines and eigenvalue multiplicities

\item[2021]
AIM workshop on Spectral graph and hypergraph theory: connections and applications
\rightloc{Online}
% 12/6/2021 Second eigenvalue multiplicity

\item
Ohio State University Mathematics Colloquium
\rightloc{Columbus, OH}
% 11/18/2021 Equiangular lines

\item
Ohio State University Combinatorics \& Probability Seminar
\rightloc{Columbus, OH}
% 11/18/2021 Second eigenvalue multiplicity of bounded degree graphs

\item 
National University of Singapore (NUS) Young Mathematician Lecture Series
\rightloc{Online} 
% 8/26/2021 Equiangular lines and eigenvalue multiplicities


\item 
Canadian Discrete and Algorithmic Math Conference (CanaDAM): \textbf{Plenary Lecture} \rightloc{Online}
% 5/ /2021

\item Workshop on Extremal and Algorithmic Aspects of Partition Functions
\rightloc{Online}
% 5/18/2021 Extremal regular graphs: independent sets, colorings, and graph homomorphisms	

\item 
Copenhagen--Jerusalem Combinatorics Seminar \rightloc{Online}
% 4/29/2021 The geometry of transitive sets, with applications to eigenbasis of Cayley graphs

\item Simons Collaboration: Algorithms \& Geometry Monthly Meeting \rightloc{Online}
%4/23/2020 Every measure on a sphere has a subgaussian basis

\item Caltech/UCLA Joint Analysis Seminar \rightloc{Online}
%1/19/2021 The joints problem for varieties

\item Joint Math Meetings MAA Invited Paper Session ``Coding Theory and Geometry''
\rightloc{Online}
%1/8/2021  The joints problem for varieties

\item[2020]
Warwick Centre for Discrete Mathematics and its Applications seminar \rightloc{Online}
% 12/8/2020 Equiangular lines, spherical two-distance sets, and spectral graph theory

\item Virtual Harmonic Analysis Seminar \rightloc{Online}
% 11/11/2020 The joints problem for varieties


\item University of Wisconsin Number Theory / Representation Theory Seminar \rightloc{Online}
% 9/10/2020 The joints problem for varieties


\item Princeton Discrete Mathematics Seminar \rightloc{Online}
% 9/3/2020 The joints problem for varieties


\item Big Seminar by Laboratory of Combinatorial and Geometric Structures \rightloc{Online}
% 8/13/2020 The joints problem for varieties

\item SCMS Combinatorics Seminar \rightloc{Online}
% 8/6/2020 Equiangular lines, spherical two-distance sets, and spectral graph theory

\item Cumberland Conference Plenary speaker (Canceled due to COVID-19)
%on Combinatorics, Graph Theory, and Computing  
\rightloc{Williamsburg, VA}
%5/23-24/2020 

%\item Workshop on Critical and Collective Effects in Graphs and Networks (CCEGN-V) \rightloc{Cape Cod, MA}
%5/18-22/2020 Canceled due to COVID-19

\item Simons Collaboration: Algorithms \& Geometry Annual Conference Plenary Lecture (Canceled due to COVID-19) \rightloc{New York, NY}
%5/15/2020 


\item Webinar in Additive Combinatorics \rightloc{Online}
% 5/18/2020 Popular differences for Roth and Szemeredi

\item Stanford Online Combinatorics Seminar \rightloc{Online}
% 4/23/2020 Regularity methods for sparse graphs and its applications


\item Stanford Math Department Colloquium \rightloc{Stanford, CA}
% 3/5/2020 Equiangular lines and eigenvalue multiplicities

\item Oberwolfach workshop: Combinatorics
\rightloc{Oberwolfach, Germany}
%1/6/2020 Equiangular lines with a fixed angle

\item[2019] Shanghai Center for Mathematical Sciences (Fudan) Discrete Math.\ Seminar
\rightloc{Shanghai, China}
%12/20 Equiangular lines and eigenvalue multiplicities


\item Conference on Graph Theory and its Applications: A Tribute to Prof. Fan Chung
\rightloc{Sanya, China}
%12/17 Equiangular lines and eigenvalue multiplicities

\item Atlanta Lectures Series in Combinatorics and Graph Theory at Emory
\rightloc{Atlanta, GA}
%11/23/2019 Equiangular lines with a fixed angle

\item Princeton Discrete Mathematics Seminar \rightloc{Princeton, NJ}
%10/17/2019 Equiangular lines with a fixed angle

\item Banff workshop: Probabilistic and Extremal Combinatorics \rightloc{Banff, AB}
%9/2/2019 Equiangular lines with a fixed angle


%%Random Structures Algorithms workshop
%7/19/2019 A reverse Sidorenko inequality

\item ETH Zurich Theory of Combinatorial Algorithms Mittagsseminar \rightloc{Z\"urich, Switzerland}
%5/23/2019 Packing superballs

\item Oberwolfach workshop: Combinatorics, Probability and Computing \rightloc{Oberwolfach, Germany}
%4/19/2019 A reverse Sidorenko inequality

\item Rutgers Discrete Math Seminar \rightloc{Piscataway, NJ}
%2/18/2019 A reverse Sidorenko inequality

\item Yale Combinatorics Seminar \rightloc{New Haven, CT}
%1/31/2019 A reverse Sidorenko inequality

\item Stanford Combinatorics Seminar \rightloc{Stanford, CA}
%1/17/2019 A reverse Sidorenko inequality

\item[2018] Clay Math Institute workshop: Recent Advances in Extremal Combinatorics \rightloc{Oxford, UK}
%12/3-7/2018 A reverse Sidorenko inequality

\item ICM satellite workshop --- Combinatorics: Extremal, Probabilistic and Additive \rightloc{S\~ao Paulo, Brazil}
%7/23--27/2018 The number of independent sets in an irregular graph

\item Simons Institute workshop: Pseudorandomness Reunion \rightloc{Berkeley, CA}
%6/19/2018 Efficient Arithmetic Regularity and Removal Lemmas for Induced Bipartite Patterns

\item MIT Workshop on Local Algorithms (WOLA 2018)  \rightloc{Cambridge, MA}
%6/10/2018 Efficient property testing of induced bipartite patterns in groups

\item MIT workshop on Sublinear Algorithms: bootcamp tutorial \rightloc{Cambridge, MA}
%6/10/2018 Graph regularity method and applications to property testing

\item SIAM Conference on Discrete Mathematics: minisymposium \rightloc{Denver, CO}
%6/6/2018 A Fast New Algorithm for Weak Graph Regularity

\item SIAM Conference on Discrete Mathematics: D\'enes K\"onig Prize Lecture \rightloc{Denver, CO}
%6/5/2018 Pseudorandom Graphs and the Green-Tao Theorem

\item Georgia Tech workshop: Algorithms and Randomness \rightloc{Atlanta, GA}
%5/14/2018 Large deviations in random graphs

\item Northeastern U.\ Network Science Institute Talk \rightloc{Boston, MA}
%5/3/2018 Large Deviations and Exponential Random Graphs

\item AMS Sectional Meeting at Northeastern University \rightloc{Boston, MA}
%4/21-22/2018 Efficient arithmetic regularity and removal lemmas for induced bipartite patterns

\item Rutgers Discrete Math Seminar \rightloc{Piscataway, NJ}
%4/16/2018 Tower-type bounds for Roth’s theorem with popular differences

\item Tsinghua YMSC minicourse \rightloc{Beijing, China}
%3/27,28,29/2018 Pseudorandom graphs and the Green-Tao theorem

\item CMU ACO Seminar \rightloc{Pittsburgh, PA}
%2/22/2018 Tower-type bounds for Roth’s theorem with popular differences

\item Harvard CMSA workshop: Probabilistic and Extremal Combinatorics \rightloc{Cambridge, MA}
% 2/5-9/2018 Efficient arithmetic regularity and removal lemmas for induced bipartite patterns

\item UCLA Combinatorics Seminar \rightloc{Los Angeles, CA}
% 1/18/2018 Tower-type bounds for Roth’s theorem with popular differences

\item[2017] Harvard CMSA workshop: Additive Combinatorics \rightloc{Cambridge, MA}
% 10/2/17 Tower-type bounds for Roth’s theorem with popular differences

\item Birmingham workshop: Interactions with Combinatorics \rightloc{Birmingham, UK}
% 6/29/17

\item BGSMath workshop: Random Discrete Structures and Beyond \rightloc{Barcelona, Spain}
% 6/9/17

\item SFSU: ACG Seminar \rightloc{San Francisco, CA}
% 5/10/17

\item Stanford Math Department Colloquium \rightloc{Stanford, CA}
% 5/4/17

\item Simons Institute workshop: Structure and Randomness \rightloc{Berkeley, CA}
% 4/10/17

\item MIT Combinatorics Seminar \rightloc{Cambridge, MA}
  % 2/22/17 Large deviations in discrete random structures

\item UC Berkeley Combinatorics Seminar
\rightloc{Berkeley, CA}
  % 1/25/17 Pseudorandomness in graph theory and combinatorics

\item Simons Institute workshop: Pseudorandomness Boot Camp \rightloc{Berkeley, CA}
  % 1/20/17 Pseudorandomness and Regularity in Graphs IV: Sparse regularity

\item Stanford Combinatorics Seminar
\rightloc{Stanford, CA}
  % 1/12/17 Large deviations for arithmetic progressions

\item Oberwolfach workshop: Combinatorics \rightloc{Oberwolfach, Germany}
  %

\item[2016] Turing Institute workshop: Large-scale structures in random graphs \rightloc{London, UK}
  % 12/15/16 Large deviations in random graphs

\item Birmingham Combinatorics Seminar \rightloc{Birmingham, UK}
  % 11/24/16 Large deviations in random graphs

\item IH\'ES Seminar \rightloc{Bures-sur-Yvette, France}
  % 11/17/16 Large deviations in random graphs

\item Warwick DIMAP Seminar \rightloc{Coventry, UK}
  % 6/21/16 Upper tails for arithmetic progressions in a random set


\item LSE/Queen Mary Colloquia in Combinatorics \rightloc{London, UK}
  % 5/11/16 Quasirandom Cayley graphs

\item Oberwolfach workshop: Combinatorics, Probability and Computing \rightloc{Oberwolfach, Germany}
  % 4/18/16 Quasirandom Cayley graphs

\item Simons Symposium: Analysis of Boolean Functions \rightloc{Kr\"un, Germany}
  % 4/6/16 Quasirandom Cayley graphs
  % 4/6/16 Pseudorandomness in the the Green-Tao theorem

\item British Mathematical Colloquium: Combinatorics Workshop \rightloc{Bristol, UK}
  % 3/22/16 Quasirandom Cayley graphs

\item Oxford Mathematical Institute North meets South Colloquium \rightloc{Oxford, UK}
  % 2/19/16 Triangles and equations

\item AMS-MAA Joint Mtgs: AMS Spec.~Session on Pseudorandomness and Its Applications \rightloc{Seattle, WA}
  % 1/8/16 Pseudorandomness in the the Green-Tao theorem

\item[2015]
  London School of Economics Discrete Mathematics and
  Game Theory Seminar \rightloc{London, UK}
  % 11/26/15 Large deviations in random graphs

\item
  Queen Mary Combinatorics Seminar \rightloc{London, UK}
  % 11/23/15 Large deviations in random graphs

\item Warwick Combinatorics Seminar \rightloc{Coventry, UK}
  % 11/20/15 Large deviations in random graphs

\item Oxford Combinatorial Theory Seminar \rightloc{Oxford, UK}
  % 11/17/15 Large deviations in random graphs
  
\item Northeastern U.\ workshop: Random Graphs, Simplicial Complexes, and their Appl'ns \rightloc{Boston, MA}
  % 5/21/15 Large deviations in random graphs

\item U.\ of Chicago Combinatorics and Theoretical Computer
  Science Seminar \rightloc{Chicago, IL}
  % 5/19/15 Large deviations in random graphs

\item Rutgers Discrete Math Seminar \rightloc{Piscataway, NJ}
  % 3/23/15 Large deviations in random graphs


\item ICERM workshop: Crystals, Quasicrystals and Random Networks
  \rightloc{Providence, RI}
  % 2/12/15 Large deviations in random graphs

\item[2014] Atlanta Lectures Series in Combinatorics and Graph
  Theory at Emory \rightloc{Atlanta, GA}
  % 11/1/2014 The Green-Tao theorem and a relative Szemer\'edi theorem

\item GSU Colloquium
  \rightloc{Atlanta, GA}
  % 10/31/2014 A century of progress on arithmetic progressions

\item CRM workshop: New Topics in Additive Combinatorics
  \rightloc{Montreal, QC}
  % 10/6/2014 The Green-Tao theorem and a relative Szemer\'edi theorem


\item IMA workshop: Additive and Analytic Combinatorics
  \rightloc{Minneapolis, MN}
  % 10/1/2014 The Green-Tao theorem and a relative Szemer\'edi theorem


\item Clay Math Institute workshop: Extremal and
  Probabilistic Combinatorics
  \rightloc{Oxford, UK}
  % 6/2/2014 Large deviations in random graphs: revisiting the infamous upper tail

\item Georgia Tech Combinatorics Seminar
  \rightloc{Atlanta, GA}
  % 4/4/2014 The Green-Tao theorem and a relative Szemer\'edi theorem

\item IAS Computer Science/Discrete Mathematics Seminar
  \rightloc{Princeton, NJ}
  % 3/3/2014 The Green-Tao theorem and a relative Szemer\'edi theorem

\item Oxford Combinatorial Theory Seminar \rightloc{Oxford, UK}
  % 1/2014 Sparse graph limits and scale-free networks

\item London School of Economics Discrete Mathematics and
  Game Theory Seminar \rightloc{London, UK}
  % 1/16/2014 The Green-Tao theorem and a relative Szemer\'edi theorem

\item Eurandom: Minicourse on Graph Limits
  \rightloc{Eindhoven, Netherlands} \\
  \mbox{}\qquad (6-hour minicourse co-taught with Christian Borgs)
  % 1/13/2014 Graph limits

\item Oberwolfach workshop: Combinatorics
  \rightloc{Oberwolfach, Germany}
  % 1/2014 The Green-Tao theorem and a relative Szemer\'edi theorem 20min

\item[2013] Simons Institute workshop: Neo-Classical Methods in
  Discrete Analysis \rightloc{Berkeley, CA}
  % 12/2013 The Green-Tao theorem and a relative Szemer\'edi theorem

\item Rutgers Discrete Math Seminar \rightloc{Piscataway,
    NJ}
  % 11/2013 The Green-Tao theorem and a relative Szemer\'edi theorem

\item MIT Combinatorics Seminar \rightloc{Cambridge, MA}
  % 10/16/13 The Green-Tao theorem and a relative Szemer\'edi theorem

\item Yale Combinatorics and Probability Seminar
  \rightloc{New Haven, CT}
  % 10/11/13 The Green-Tao theorem and a relative Szemer\'edi theorem

\item Microsoft Research Theory Reading Group \rightloc{Cambridge, MA}
  % 6/28/2013 Green-Tao, Szemer\'edi, and transference

\item Oberwolfach workshop: Combinatorics and Probability
  \rightloc{Oberwolfach, Germany}
  % 1/2013 large deviations in random graphs

\item[2012] MIT Combinatorics Seminar \rightloc{Cambridge, MA}   % sparse graph regularity
\item SIAM Conference on Discrete Mathematics
  \rightloc{Halifax, NS}   % sparse graph regularity
% \item[01/2011] AMS/MAA Joint Math Meetings, Undergrad Poster Session, {\em prize winner} \rightloc{New Orleans, LA}
% \item[01/2010] AMS/MAA Joint Math Meetings, Undergrad Poster Session, {\em prize winner} \rightloc{San Francisco, CA}
% \item[01/2010] AMS/MAA Joint Math Meetings Contributed Session \rightloc{San Francisco, CA}
\item[2009] MIT Combinatorics Seminar \rightloc{Cambridge, MA}
  %more sums than differences sets
\end{itemize}

%\section*{Workshop attendence (selected)}
%
%\begin{itemize}[leftmargin=.4in,itemsep=5pt,label={}]
%
%\item[2014]
%\item Clay Math Institute: Extremal and Probabilistic Combinatorics
%\item Oberwolfach: Combinatorics \rightloc{Oberwolfach, Germany}
%
%\item[2013] Simons Institute: Neo-classical Methods in Discrete Analysis
%  \rightloc{Berkeley, CA}
%\item Banff: Geometric and Topological Graph Theory
%  \rightloc{Banff, AB}
%\item Oberwolfach: Combinatorics and Probability
%  \rightloc{Oberwolfach, Germany}
%\item UCLA/IPAM: Extremal and Probabilistic Combinatorics
%  \rightloc{Los Angeles, CA}
%\item Oberwolfach: Graph Theory \rightloc{Oberwolfach, Germany}
%\end{itemize}




\section*{Teaching}
%{\footnotesize [U = Undergraduate, G = Graduate]}

\textbf{Graph Theory and Additive Combinatorics} (graduate, MIT) 

\emph{Lecture videos available through MIT OpenCourseWare. Links on \url{https://yufeizhao.com/gtacbook/}}

\qquad 
\begin{tabular}[l]{p{2cm}p{3cm}p{3.5cm}}
\toprule
Term & Enrollment \newline (credit + listener) & Instructor evaluation \newline  (max 7) \\
\midrule
Fall 2021 & 36 +  9 & 6.7 \\
Fall 2019 & 30 + 14 & 6.9 \\
Fall 2017 & 31 + 17 & 7.0 \\
\bottomrule
\end{tabular}

\medskip


\textbf{Probabilistic Methods in Combinatorics} (graduate, MIT) 

\qquad 
\begin{tabular}[l]{p{2cm}p{3cm}p{3.5cm}}
\toprule
Fall 2020 & 25 + 16 & {\footnotesize Not rated due to COVID} \\
Spring 2019 & 47 + 25 & 6.9 \\
\bottomrule
\end{tabular}

\medskip

\textbf{Combinatorial Analysis} (undergraduate, MIT) 

\qquad 
\begin{tabular}[l]{p{2cm}p{3cm}p{3.5cm}}
\toprule
Fall 2018 & 22 + 7 & 6.8 \\
\bottomrule
\end{tabular}

\medskip

\textbf{Mathematical Problem Solving: Putnam Seminar} (undergrad first-year seminar, MIT) 

\qquad 
\begin{tabular}[l]{p{2cm}p{3cm}p{3.5cm}}
\toprule
Fall 2021 & 26 & 6.7 \\
Fall 2020 & 16 & {\footnotesize Not rated due to COVID} \\
Fall 2019 & 22 & 6.3 \\
Fall 2018 & 22 & 6.4 \\
Fall 2017 & 10 & 6.3 \\
\bottomrule
\end{tabular}


\textbf{Previous teaching:}

\begin{itemize}[topsep=-.5ex,label={},leftmargin=.2in]


\item Polynomial Method in Combinatorics, graduate-level, Oxford, 2016

\item Undergraduate tutorials in geometry, Oxford, 2016
\end{itemize}

%\begin{longtable}[l]{lrcl}
%
%\textbf{MIT} 	
%	& Fall 2019 & G & 18.217 Graph Theory and Additive Combinatorics \\
%	& Fall 2019   & U & 18.A34 Mathematical Problem Solving (Putnam Seminar) \\
%	& Spr 2019 & G & 18.218 The Probabilistic Method \\
%	& Fall 2018   & U & 18.A34 Mathematical Problem Solving (Putnam Seminar) \\
%	& 		    & U & 18.211 Combinatorial Analysis \\
%	& Fall 2017   & U & 18.A34 Mathematical Problem Solving (Putnam Seminar) \\
%   	&		    & G & 18.S997 Graph Theory and Additive Combinatorics \\
%%	& Spr 2013 & U & 18.03: Differential Equations (recitation) \\
%\textbf{Oxford}	& MT 2016  & U & Geometry (tutorial) \\
%	& TT 2016  & G & Polynomial Method in Combinatorics
%\end{longtable}


\newpage




\section*{Advising and mentorship}

Current PhD students: 
\begin{itemize}[topsep=-.5ex,label={},leftmargin=.2in]
\item Aaron Berger
\item Dingding Dong (Harvard)
\item Ashwin Sah
\item Mehtaab Sawhney
\end{itemize}

Former PhD students:
\begin{itemize}[topsep=-.5ex,label={},leftmargin=.2in]
	\item Benjamin Gunby (Harvard PhD '21) $\to$ Rutgers postdoc
	\item Jonathan Tidor (MIT PhD '22) $\to$ Stanford Science Fellow (postdoc)
\end{itemize}

Undergraduate research supervised: 
\begin{itemize}[topsep=-.5ex,label={},leftmargin=.2in]
\item Yang Liu (2018) $\to$ Stanford PhD student
\item Ryan Alweiss (2018) $\to$ Princeton PhD student
\item Yunkun Zhou (2018--2019) $\to$ Stanford PhD student
\item Mehtaab Sawhney (2018--2020), \emph{Morgan Prize winner}  $\to$ MIT PhD student
\item Ashwin Sah (2018--2020), \emph{Morgan Prize winner} $\to$ MIT PhD student
\item David Stoner (2018--2019), \emph{Morgan Prize honorable mention} $\to$ Stanford PhD student
\item Yuan Yao (2019--2020) $\to$ MIT PhD student
\item Shengtong Zhang (2019--2021) $\to$ Stanford PhD student
\item Hung-Hsun Yu (2019--2021) $\to$ Cambridge Part III \& Princeton PhD student
\item Mihir Singhal (2019)
\item Zachary Chroman (2019) $\to$ Cambridge Part III
\item Carl Schildkraut (2020-- )
\item Milan Haiman (2020--2021)
\item Anqi Li (2021)
\item Dain Kim (2021)
\item Mingyang Deng (2022-- )
\item Tomasz Slusarczyk (2022-- )
\item Aleksandre Saatashvili (2022-- )
\item Saba Lepsveridze (2022-- )
\end{itemize}

Postdoctoral researchers mentored:
\begin{itemize}[topsep=-.5ex,label={},leftmargin=.2in]
\item Zilin Jiang (2018--2020) $\to$ Assistant Professor at Arizona State University
\item L\'aszl\'o Mikl\'os Lov\'asz (2018--2020) $\to$ working in industry
\end{itemize}

\section*{Service}

Undergraduate first-year advising, Fall 2017---current

Co-organizer of MIT Combinatorics Seminar, Fall 2017---current

Organizer of the MIT team for the Putnam Competition, Fall 2017---current

AMS--Simons Travel Grants Committee Member, 2021---2024

Springer Graduate Texts in Mathematics (GTM) Advisory Board Member, 2022---2024
% Prague Summer School on Discrete Mathematics committee

%Served as referee/reviewer for:
%ACM Symposium on Theory of Computing (STOC),
%ACM-SIAM Symposium on Discrete Algorithms (SODA), 
%Advances in Mathematics, 
%Combinatorics, Probability and Computing,
%Discrete Analysis,
%Discrete Mathematics,
%Duke Journal of Mathematics,
%Electronic Communications in Probability,
%Geometric and Functional Analysis,
%Forum of Mathematics Pi, 
%Forum of Mathematics Sigma, 
%International Mathematics Research Notices, 
%Journal of the London Mathematical Society,
%Proceedings of the London Mathematical Society,
%Journal of Combinatorial Theory Series A, 
%Journal of Combinatorial Theory Series B,
%Probability Theory and Related Fields,
%Random Structures and Algorithms,
%SIAM Journal on Discrete Mathematics,
%and other journals and conferences.



\section*{Other Experiences and Activities}

Organizer and Chief Coordinator of Cyberspace Mathematical Competition (CMC) 2020

Quantitative Research Intern, D.\ E.\ Shaw \& Co., New York

% MIT PRIMES Mentor --- 2013--2015 (Lusztig PRIMES Mentor in 2015)

MIT Lusztig PRIMES Mentor

Research Experience for Undergraduates at Duluth participant (mentor: Joe Gallian) %--- Summer 2009

Deputy Leader for Canadian IMO Team %--- 2008

Instructor at Canadian IMO Training Camps %--- various summers and winters

Mentor at AwesomeMath Summer Program, Dallas %--- Summer 2007

Trainer at US Math Olympiad Summer Program, Lincoln, Nebraska %--- Summer 2007

Teacher at Spirit of Math Schools, Toronto %--- 2005--2006

\vspace{\fill}

{\hfill \footnotesize CV updated: \today}

\end{document}
