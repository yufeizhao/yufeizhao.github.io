\documentclass{article}

\usepackage{amsmath,amssymb,hyperref}
\newcommand{\FF}{\mathbb{F}}
\newcommand{\ZZ}{\mathbb{Z}}
\newcommand{\NN}{\mathbb{N}}
\newcommand{\RR}{\mathbb{R}}
\newcommand{\CC}{\mathbb{C}}

\begin{document}

\begin{center}
Homework exercises for \\
\textbf{Polynomial Method in Combinatorics}
\end{center}
	
\begin{enumerate}

\item

Consider the map $\gamma \colon \RR \to \RR^2$ given by
\[
\gamma(t) = (t^{17} + t^5 + 3, t^{14} +t^9 - 7t^2 -1).
\]
Prove that there is some non-zero polynomial $P(x, y)$ so that image of
$\gamma$ is contained in the zero-set of $P$. Can you give some estimate for the degree of $P$?

\item

Show that for any $N$ lines in $\RR^3$, there is some non-zero polynomial of degree $O(N^{1/2})$ that vanishes on all lines.

State and prove a similar result for $k$-planes in $\RR^n$ for any dimensions $k \le n$.


\item

(Schwarz--Zippel lemma) Let $A_i \subset \FF$ be finite subsets with $|A_i| = N$ for each $i = 1, \dots, n$. Let $P$ be a non-zero polynomial over $\FF$ on $n$ variables and total degree at most $D$. Show that the number of zeros of $P$ in $A_1 \times \dots \times A_n$ is at most $DN^{n-1}$.

In particular, a nonzero polynomial of degree $D$ vanishes on at most $D/|\FF|$ fraction of points of $\FF^n$.


\item

(Joints problem in higher dimensions) Let $\mathcal{L}$ be a set of $L$ lines in $\RR^n$. A \emph{joint} of $\mathcal{L}$ is defined to be a point that lies in $n$ lines of $\mathcal{L}$ pointing in linearly-independent directons.

By extending the argument shown in class, prove that a set of $L$ lines in $\RR^n$ determines at most $C_n L^{n/(n-1)}$ joints.

\item

(Joints of axis parallel lines/Loomis-Whitney theorem) Let $\mathcal{L}_i$ be a set of $L_i$ lines parallel to the $x_i$-axis in $\RR^n$. Let $\mathcal{L} = \bigcup_i \mathcal{L}_i$. Show that the number of joints in $\mathcal{L}$ is at most $\prod_{i=1}^n L_i^{1/(n-1)}$ (note that there is no extra constant).


It is a good idea to start with the $n=3$ case.

\item

Let $A,B,C \subset \RR$ with $|A|<|B|<|C|$. Let $P(x,y,z) = \prod_{a \in A} (x-a)$. Prove that $P$ is a minimum degree polynomial that vanishes on the grid $A \times B\times C$, and furthermore every minimum degree polynoial vanishing on the grid is a multiple of $P$.

This example illustrates how the minimum degree polynomial picks up the most important lines in the proof of the joints problem.

What happens when $|A| = |B| = |C|$?


\item

Complete the proof of Wolff's hairbrush method bound: if $\ell_1, \dots, \ell_M$ are lines in $\FF_q$, and suppose that at most $q+1$ of the lines lie in any plane, then their union has cardinality $\gtrsim q^{3/2}M^{1/2}$ (in particular this implies that if $K \subset \FF_q^n$ is a Kakeya set, then $|K| \ge (1/2)|q|^{(n+2)/2}$).

The proof uses the \emph{hairbrush method}. A \emph{hairbrush} is a set of lines $\ell_j$ meeting a fixed $\ell_i$ (but not including $\ell_i$ itself). Show there exists a hairbrush containing $(1/2)q^2M/|X|$ lines.

\item

Verify the following properties of the Hermitian variety $H$ : $x^{p+1} + y^{p+1} + z^{p+1} = 1$ in $\FF_q^3$ with $q = p^2$:

(a) $H$ contains $\Theta(q^{5/2})$ points

(b) $H$ contains $\Theta(q^2)$ lines, with at most $O(q^{1/2})$ lines in every 2-plane.

Also, show that the following analogue of the unitary group on $\FF_q^n$ acts transitively on $H$:
\[U(\FF_q^n) := \{ g \in \mathrm{GL}(\FF_q^n) : \langle v,w \rangle = \langle
gv, g \rangle \text{ for all } v,w \in \FF_q^n \},
\]
where the inner product is defined by
\[
\langle v,w\rangle := v_1 \overline{w_1} + v_2 \overline{w_2} +  v_3 \overline{w_3}
\]
where $\overline{x} := x^p$ is conjugation in $\FF_q$.

\item
(Heisenberg surface \href{http://arxiv.org/abs/math/0204234}{{[Mockenhaupt--Tao]}}) Let $p$ be a prime. Let $X \subset \FF_{p^2}^3$ be the surface defined by the equation $x - x^p + yz^p - zy^p = 0$. Show that $X$ is a set of $\Theta(p^5)$ points, and contains $p^4$ lines, with no more than $p$ lying on any plane.

\item

Let $Q \subset \FF_{q}^4$ be the degree 2 hypersurface defined by the equation $x_1^2 + x_2^2 - x_3^2 - x_4^2 = 1$. Prove that each point $x \in Q$ lies in $\Theta(q)$ lines in $Q$, and all of these lines lie in a 3-plane. Furthermore, check that $Q$ contains $\Theta(q^3)$ points and $\Theta(q^3)$ lines.

\item

Let $W$ be a subspace of functions $\FF^n \to \FF$ that satisfies the degree $D$ vanishing lemma, i.e., whenever $f \in W$ vanishes at $D+1$ points on a line, then $f$ vanishes at every point on the line.

Show that if $|\FF|\ge D+1$, then $\dim W \le (D+1)^n$.

(Open) What is the maximum possible $\dim W$ (as a function of $n$ and $D$)?

\item

Prove the divisibility lemma: if $P(x,y)$ and $Q(x)$ are polynomials such that $P(x,Q(x))$ is the zero polynomial, then $P(x,y) = (y-Q(x))S(x,y)$ for some polynomial $S(x,y)$.

\item

Let $D < q$. Show that for any function $g \colon \{0, \dots, D\}^n \to \FF_q$, there is a unique polynomial $P \colon \FF_q^n \to \FF_q$ so that $P = g$ on $\{0, \dots, D\}^n$ and $P$ has degree at most $D$ in every single variable.

\item

Let $N \subset \FF_q^n$ be a Nikodym set, i.e., for every $x \in \FF_q^n$, there is a line $\ell \ni x$ with $\ell \setminus \{x\} \subset N$. Let $P \colon \FF_q^n \to \FF_q$ be a polynomial of degree at most $D < (q-1)/n$ in each variable. Show that if we know $P$ on a Nikodym set $N$, then we can recover $P$ everywhere. Combine this observation with the previous exercise to show that $|N| \ge c_n q^n$.

\item

Show that the following two versions of Szemerédi--Trotter theorem are equivalent:

\begin{itemize}
	\item The number of incidences between any $S$ points and $L$ lines in the plane is $O(S^{2/3}L^{2/3} + S +L)$
	\item The number of $r$-rich points among any $L$ lines in the plane is $O(L^2/r^3 + L/r)$
\end{itemize}

\item
Show that the $O(L^2/r^3 + L/r)$ bound on the number of $r$-rich points cannot be improved by verifying the construction suggested in lecture: the $N \times N$ square grid of points and $r$ different slopes with rational coordinates of small numerators and denominators.

\item

(Harnack inequality) Show that if $P(x,y)$ is a nonzero polynomial of degree $D$ in two variables, then $\RR^2 \setminus Z(P)$ contains $O(D^2)$ connected components.

\emph{Hint:} Use Bezout's theorem to bound the number of "unbounded regions" by considering intersections with a large circle. For bounded regions, note that $P$ must contain a critical point (either a maximum or a minimum) in each bounded region. Analyze the number of critical points of $P(x,y)$ (where both partial derivatives vanish) using Bezout's theorem. (Be careful if the two partial derivatives share a common factor, in which case you can perturb $P$ slightly to remove this issue.)

\item

Prove the ham sandwich theorem using the Borsuk--Ulam theorem, stated below.

Borsuk--Ulam theorem: If $\phi \colon S^N \to \RR^N$ is continuous and antipodal (i.e., $\phi(-x) = -\phi(x)$ for all $x$), then the image of $\phi$ contains the origin.

\item

(Unit distance problem)

\begin{enumerate}
	\item[(a)] Prove that the number of incidences between $N$ unit circles and $S$ points in the plane is $O(N^{2/3} S^{2/3} + N + S)$. (Hint: use polynomial partitioning)
	\item[(b)] As a corollary, show that a set of $N$ points in the plane determines $O(N^{4/3})$ unit distances.  
(This is currently the best known bound on the unit distance problem. The truth is conjectured to be $N^{1+o(1)}$.)
	\item[(c)] Construct a set of $N$ parabolas of the form $y = (x-a)^2 + b$ and $N$ points in the plane so that the number of incidences is $\Theta(N^{4/3})$.  
This example shows that it is hard to improve the bound on the unit distance problem as it is difficult to distinguish between unit circles and unit parabolas.
\end{enumerate}

\item
\begin{enumerate}
\item[(a)] Prove that the number of incidences between $N$ circles and $S$ points in the plane is $O(S^3 + N)$ (this is an "easy bound").
\item[(b)] Use polynomial partitioning to improve the bound estimate to $O(S^{3/5} N^{4/5} + N + S))$
\item[(c)] As a corollary, show that the number of $r$-rich points is $O(N^2 r^{-5/2})$ (points that are contained in $\ge r$ circles).
\end{enumerate}


\item

Let $P$ be a set of $N$ points in the plane with $\epsilon N$ distinct distances. Show that $P$ has an $r$-rich partial symmetry with $r \ge e^{c\epsilon^{-1}}$ for some $c > 0$.

\item

(The square grid example)
\begin{enumerate}
\item[(a)] Let $G_0$ denote the set of points in $\RR^3$ of the form $(a,b,0)$ with $a,b$ positive integers up to $L^{1/4}$, and $G_1$ the set of points $(a,b,1)$ with $a,b$ in the same range. Let $\mathcal{L}$ be the set of lines containing one point of $G_0$ and one point of $G_1$. Show that the number of $r$-rich points in $\mathcal{L}$ is $\gtrsim L^{3/2}r^{-2}$ for all $2 \le r \le (1/400) L^{1/2}$.
\item[(b)] Let $P$ be a square grid of $N$ points in the plane. Show that the number of $r$-rich partial symmetries of $P$ is $\Theta(N^3r^{-2})$ for all $2 \le r \le N/400$.
\end{enumerate}

\item

Let $\FF$ be an infinite field. Let $N \ge \binom{n+d}{d}$. Prove that there exists a set of $N$ points in $\FF^n$ with degree greater than $D$.

\end{enumerate}
\end{document}
