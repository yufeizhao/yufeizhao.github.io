\documentclass[11pt]{amsart}
\usepackage{amsmath, fancyhdr,multicol, etaremune, lastpage, xcolor, url,enumitem,graphics,titlesec, tabularx,longtable, booktabs}
\usepackage{microtype}
\usepackage[hidelinks]{hyperref}
\usepackage[left=.9in, right=.9in, top=.9in, bottom=.9in]{geometry}

\usepackage[T1]{fontenc}
\usepackage[bitstream-charter]{mathdesign}

\definecolor{darkblue}{rgb}{0,0,.6}
\newcommand{\blue}{\color{darkblue}}
\setlength{\parindent}{0in}


\pagestyle{fancy}

\lhead{\blue \sc Yufei Zhao}
%\leftmark
%\chead{\blue Yufei Zhao}
\rhead{\blue {\thepage}/\pageref{LastPage}}
%\rfoot{\tiny Updated: \today}
%\rightmark
%\renewcommand{\headrulewidth}{0pt}

\cfoot{}

%Section header formating
%\renewcommand{\section*}[1]{{\vspace{14pt}\blue \bfseries \large #1 \par \vspace{7pt}}}

\titleformat{\section}{\normalfont\bfseries\large\blue}{\thesection}{1em}{}
\titlespacing*{\section}{0em}{1em}{.2em}
\setlength{\parskip}{.4em}
\setlength{\extrarowheight}{.4em}
\setlength{\lineskip}{.2em}


% \newenvironment{newlist}{\begin{list}{$\bullet$} {\setlength{\leftmargin}{1.5em}}}{\end{list}}

% To add some paragraph space between lines.
% This also tells LaTeX to preferably break a page on one of these gaps
% if there is a needed pagebreak nearby.
\newcommand{\blankline}{\quad\vspace{-6pt}\pagebreak[2]}




\newcommand{\rightloc}[1]{\hfill {\raggedright #1}}
\newcommand{\rightdate}[1]{\hfill {\raggedright #1}}
\renewcommand{\labelenumi}{\theenumi.}

\newcommand{\arXiv}[1]{\href{http://arxiv.org/abs/#1}{\tt arXiv:#1}}

\newcommand{\p}[1]{{\bfseries #1}}
\renewcommand{\j}[1]{{\frenchspacing\itshape #1}}
\newcommand{\q}{\null\quad}

\begin{document}

\thispagestyle{empty}


\parbox{.5\textwidth}{\scalebox{3}{\sffamily\bfseries Yufei Zhao}}
\hfill
\parbox{.5\textwidth}
{  \begin{flushright} \small
\url{http://yufeizhao.com} \\
\url{yufeiz@mit.edu} 
\\[2pt] \footnotesize
MIT Department of Mathematics \\
77 Massachusetts Ave, Room 2-271 \\  
Cambridge, MA 02139, USA
\end{flushright}
}

\vspace{12pt}

{\blue
\hrule
\vspace{0.02in}
\hrule
%\vspace{0.1in}
}

\section*{Current Position}

\p{Massachusetts Institute of Technology} \rightloc{Cambridge, MA} \\
\null\quad Assistant Professor, Department of Mathematics \rightdate{2017---\phantom{2020}}

\section*{Previous Positions}

\textbf{UC Berkeley} \rightloc{Berkeley, CA} \\
\q Simons Institute Research Fellow \rightdate{Spring 2017}

\p{University of Oxford} \rightloc{Oxford, UK} \\
\q Esm\'ee Fairbairn Junior Research Fellow in Mathematics, New College  \rightdate{2015---2017}

\section*{Education}

\p{Massachusetts Institute of Technology} \rightloc{Cambridge, MA}\\
\q Ph.D.~Mathematics. Advisor: Jacob Fox
\rightdate{2011---2015}

\p{University of Cambridge} \rightloc{Cambridge, UK} \\
\q M.A.St.~Mathematics with Distinction \rightdate{2010---2011}

\p{Massachusetts Institute of Technology} \rightloc{Cambridge, MA} \\
\q S.B.~Mathematics, with minor in Economics. \rightdate{2006---2010} \\
\q S.B.~Computer Science and Engineering. % GPA: $5.0/5.0$.

\section*{Research Interests}

Extremal/probabilistic/additive combinatorics;
graph theory and graph limits;
sphere packing

\section*{Selected Awards and Honors}

Johnson Prize, MIT Mathematics Department, 2015

%\quad {\small in recognition of an outstanding mathematics paper accepted for publication in a major journal}

Microsoft Research PhD Fellowship, 2013--2015

MIT Akamai Presidential Fellowship, 2011--2012

Leslie Walshaw Prize, Examination Prize, and Senior Scholarship,
Trinity College, Cambridge, 2011

% \quad {\small for top results in Cambridge Part III Mathematics examinations}

Morgan Prize Honorable Mention, 2011

%\quad {\small for outstanding research in mathematics by an undergraduate student}

Gates Cambridge Scholarship, 2010--2011

%\blankline

%Trinity College Cambridge Studentship in Mathematics, 2010

% \blankline

% Julie Payette---NSERC Research Scholarship (declined), 2010

MIT Jon A.~Bucsela Prize in Mathematics, 2010

%\quad {\small awarded to the top graduating senior in the MIT Mathematics Department}

%William Lowell Putnam Mathematics Competition

Putnam Math Competition: Three-time Putnam Fellow (top five rank) 2006, 2008, 2009; 7th Place 2007
%\item Member of MIT Team: First Place 2009, Third Place 2007 and 2008

% \blankline

% MIT 6.370 BattleCode AI Programming Competition: first place team 2008

International Mathematical Olympiad: Gold Medal 2005; Silver Medal
2006; Bronze Medal 2004

% \begin{newlist}
% \item Silver Medal, 2006  \rightloc{Ljubljana, Slovenia}
% \item Gold Medal, 2005    \rightloc{Merida, Mexico}
% \item Bronze Medal, 2004  \rightloc{Athens, Greece}
% \end{newlist}

%\blankline
%
%USA Mathematical Olympiad: Third Place 2005 and 2006
%
%\blankline
%
%Canadian Mathematical Olympiad: First Place 2004

%\blankline
%
%\textbf{Asian Pacific Mathematical Olympiad}
%
%\begin{newlist}
%\item Perfect score, 2006
%\end{newlist}

% \blankline
% Member of honor societies: Phi Beta Kappa, Eta Kappa Nu

\section*{Research Internships}

\p{Microsoft Research} New England \rightloc{Cambridge, MA} \\
\q Mentor: Henry Cohn \rightdate{Summers 2010, 2011, 2013, 2014}

\blankline

\p{Microsoft Research} Theory Group
\rightloc{Redmond, WA} \\
\q Mentor: Eyal Lubetzky \rightdate{Summer 2012}

% \blankline

% NSF Research Experience For Undergraduates (REU) at Duluth
% \rightloc{Duluth, MN}

% \quad Mentor: Joseph Gallian  \rightdate{Summer 2009}

% \begin{newlist}
% 	\item Conducted independent research on additive combinatorics and graph theory at University of Minnesota Duluth REU under the supervision of Prof.~Joseph Gallian.
% 	\item Wrote five papers intended for publication in professional research journals.
% \end{newlist}

% \blankline

% MIT Undergraduate Research Opportunities Program (UROP)
% \rightloc{Cambridge, MA}

% \quad Mentors: Richard Stanley, Michel Goemans \rightdate{2006--2007, 2009}

% \p{MIT Mathematics Department UROP} \hfill Cambridge, MA

% {\em Undergraduate Researcher} \rightdate{10/2006---5/2007, 2/2009---12/2009}

% \begin{newlist}
% 	\item Conducted research as part of the MIT Undergraduate Research Opportunities Program (UROP), under the guidance of Prof.~Richard Stanley and Prof.~Michel Goemans.
% \end{newlist}

\section*{Papers}

\begin{etaremune}[leftmargin=0.3in,itemsep=4pt,topsep=0pt,partopsep=0pt,parsep=0pt]

\item Y.~Zhao, Group representations that resist worst-case sampling, \arXiv{1705.04675}.

\item Y.~Zhao, Extremal regular graphs: independent sets and graph homomorphisms, \\
  \j{Amer. Math. Monthly}, to appear.
  %preprint \arXiv{1610.09210}.

\item B.~B.~Bhattacharya, S.~Ganguly, X.~Shao, and Y.~Zhao, \\ 
  Upper tails for arithmetic progressions in a random set, 
  preprint \arXiv{1605.02994}.

\item J.~Fox, L.~M.~Lov\'asz, Y.~Zhao,
  On regularity lemmas and their algorithmic applications, \\
  \j{Combin. Probab. Comput.}, to appear.
%  preprint \arXiv{1604.00733}.
	
\item D.~Conlon and Y.~Zhao,
  Quasirandom Cayley graphs, \\
  \j{Discrete Analysis} 2017:6, 14 pp.
  %preprint \arXiv{1603.03025}.

\item B.~B.~Bhattacharya, S.~Ganguly, E.~Lubetzky, and Y.~Zhao, \\
  Upper tails and independence polynomials in random graphs,
  \\
  \j{Adv. Math.}, to appear. 
%  preprint \arXiv{1507.04074}.

\item L.~M.~Lov\'asz and Y.~Zhao,
  On derivatives of graphon parameters, \\
  \j{J. Combin. Theory Ser. A} 145 (2017), 364--368.
%  preprint \arXiv{1505.07448}.

\item Y.~Zhao,
  On the lower tail variational problem for random graphs, \\
  \j{Combin. Probab. Comput.} 26 (2017), 301--320.
%  preprint \arXiv{1502.00867}.

\item C.~Borgs, J.~T.~Chayes, H.~Cohn, and Y.~Zhao, \\
  An $L^p$ theory of sparse graph convergence II:
  LD convergence, quotients, and right convergence, 
  \\
  \j{Ann. Probab.}, to appear.
%  preprint \arXiv{1408.0744}.

\item D.~Conlon, J.~Fox, and Y.~Zhao,
  The Green-Tao theorem: an exposition, \\
  \j{EMS Surv.~Math.~Sci.} 1 (2014), 249--282.
  % Preprint \arXiv{1403.2957}.

\item E.~Lubetzky and Y.~Zhao,
  On the variational problem for upper tails in sparse random graphs, \\
  \j{Random Structures Algorithms} 50 (2017), 420--436.
  % \arXiv{1402.6011}

\item C.~Borgs, J.~T.~Chayes, H.~Cohn, and Y.~Zhao, 
  An $L^p$ theory of sparse graph convergence I: limits, sparse random
  graph models, and power law distributions,
  preprint \arXiv{1401.2906}.

\item Y.~Zhao,
  An arithmetic transference proof of a relative Szemer\'edi theorem, \\
  \j{Math.~Proc.~Cambridge Philos.~Soc.} 156 (2014), 255--261.
  % \arXiv{1307.4959}.

\item J.~Fox and Y.~Zhao,
  A short proof of the multidimensional Szemer\'edi theorem in the primes, \\
  \j{Amer.~J.~Math.} 137 (2015), 1139--1145.
  %Preprint \arXiv{1307.4679}.

\item D.~Conlon, J.~Fox, and Y.~Zhao,
  A relative Szemer\'edi theorem,  \\
  \j{Geom. Funct. Anal.} 25 (2015), 733--762.
  %Preprint \arXiv{1305.5440}.
  %With companion note: Linear forms from the Gowers uniformity norm,
  %\arXiv{1305.5565}.

\item Y.~Zhao,
  Hypergraph limits: a regularity approach, \\
  \j{Random Structures Algorithms} 47 (2015), 205--226. %Preprint
  % \arXiv{1302.1634}.

\item H.~Cohn and Y.~Zhao,
  Sphere packing bounds via spherical codes, \\
  \j{Duke Math.~J.} 163 (2014), 1965--2002.
  %  Preprint \arXiv{1212.5966}.

\item H.~Cohn and Y.~Zhao,
  Universally optimal error-correcting codes, \\
  \j{IEEE Trans.~Inform.~Theory} 60 (2014), 7442--7450.
  %Preprint \arXiv{1212.1913}

\item E.~Lubetzky and Y.~Zhao,
  On replica symmetry of large deviations in random graphs, \\
  \j{Random Structures Algorithms} 47 (2015) 109--146. %\arXiv{1210.7013}.

\item J.~Fox, P.~Loh, and Y.~Zhao,
  The critical window for the classical Ramsey-Tur\'an problem, \\
  \j{Combinatorica} 35 (2015) 435--476. %\arXiv{1208.3276}.

\item D.~Conlon, J.~Fox, and Y.~Zhao,
  Extremal results in sparse pseudorandom graphs, \\
  \j{Adv.~Math.} {256} (2014), 206--290.
    %Preprint \arXiv{1204.6645}.

\item Y.~Zhao,
  {The bipartite swapping trick on graph homomorphisms}, \\
  \j{SIAM J.~Discrete Math.} 25 (2011), 660--680.

\item Y.~Zhao,
  {Sets characterized by the number of missing sums and differences}, \\
  \j{J.~Number Theory} 11 (2011), 2107--2134.

\item D.~Galvin and Y.~Zhao,
  {The number of independent sets in graphs with small maximum degree}, \\
  \j{Graphs Combin.} 27 (2011), 177--186.

\item Y.~Zhao,
  {Counting MSTD sets in finite abelian groups}, \\
  \j{J.~Number Theory} {130} (2010), 2308--2322.

\item Y.~Zhao,
  {Constructing numerical semigroups of a given genus},\\
  \j{Semigroup Forum} {80} (2010), 242--254.
\item Y.~Zhao,
  {Constructing MSTD sets using bidirectonal ballot sequences},\\
  \j{J.~Number Theory} {130} (2010), 1212--1220.

\item Y.~Zhao,
  {The number of independent sets in a regular graph}, \\
  \j{Combin.~Probab.~Comput.} {19} (2010), 315--320.

\item Y.~Zhao,
  {The coefficients of a truncated Fibonacci power series}, \\
  \j{Fibonacci Quart.} {46/47} (2009), 53--55.
  
\end{etaremune}

\section*{Invited Talks}

\begin{itemize}[leftmargin=.4in,itemsep=5pt,topsep=0pt,label={}]

\item[2017] Harvard CMSA workshop: Additive Combinatorics and Applications \rightloc{Cambridge, MA}
% 10/2-6/17

\item Birmingham workshop: Interactions with Combinatorics \rightloc{Birmingham, UK}
% 6/29/17

\item BGSMath workshop: Random Discrete Structures and Beyond \rightloc{Barcelona, Spain}
% 6/9/17

\item SFSU: ACG Seminar \rightloc{San Francisco, CA}
% 5/10/17

\item Stanford Math Department Colloquium \rightloc{Stanford, CA}
% 5/4/17

\item Simons Institute workshop: Structure and Randomness \rightloc{Berkeley, CA}
% 4/10/17

\item MIT Combinatorics Seminar \rightloc{Cambridge, MA}
  % 2/22/17 Large deviations in discrete random structures
  
\item UC Berkeley Combinatorics Seminar
\rightloc{Berkeley, CA}
  % 1/25/17 Pseudorandomness in graph theory and combinatorics

\item Simons Institute workshop: Pseudorandomness Boot Camp \rightloc{Berkeley, CA}
  % 1/20/17 Pseudorandomness and Regularity in Graphs IV: Sparse regularity

\item Stanford Combinatorics Seminar
\rightloc{Stanford, CA}
  % 1/12/17 Large deviations for arithmetic progressions

\item Oberwolfach workshop: Combinatorics \rightloc{Oberwolfach, Germany}
  % 

\item[2016] Turing Institute workshop: Large-scale structures in random graphs \rightloc{London, UK}
  % 12/15/16 Large deviations in random graphs

\item Birmingham Combinatorics Seminar \rightloc{Birmingham, UK}
  % 11/24/16 Large deviations in random graphs

\item IH\'ES Seminar \rightloc{Bures-sur-Yvette, France}
  % 11/17/16 Large deviations in random graphs

\item Warwick DIMAP Seminar \rightloc{Coventry, UK}
  % 6/21/16 Upper tails for arithmetic progressions in a random set


\item LSE/Queen Mary Colloquia in Combinatorics \rightloc{London, UK}
  % 5/11/16 Quasirandom Cayley graphs

\item Oberwolfach workshop: Combinatorics and Probability \rightloc{Oberwolfach, Germany}
  % 4/18/16 Quasirandom Cayley graphs 

\item Simons Symposium: Analysis of Boolean Functions \rightloc{Kr\"un, Germany}
  % 4/6/16 Quasirandom Cayley graphs
  % 4/6/16 Pseudorandomness in the the Green-Tao theorem

\item British Mathematical Colloquium: Combinatorics Workshop \rightloc{Bristol, UK}
  % 3/22/16 Quasirandom Cayley graphs
  
\item Oxford Mathematical Institute North meets South Colloquium \rightloc{Oxford, UK}
  % 2/19/16 Triangles and equations

\item AMS-MAA Joint Mtgs: AMS Spec.~Session on Pseudorandomness and Its Applications \rightloc{Seattle, WA}
  % 1/8/16 Pseudorandomness in the the Green-Tao theorem

\item[2015] 
  London School of Economics Discrete Mathematics and
  Game Theory Seminar \rightloc{London, UK}
  % 11/26/15 Large deviations in random graphs

\item 
  Queen Mary Combinatorics Seminar \rightloc{London, UK}
  % 11/23/15 Large deviations in random graphs

\item Warwick Combinatorics Seminar \rightloc{Coventry, UK}
  % 11/20/15 Large deviations in random graphs

\item Oxford Combinatorial Theory Seminar \rightloc{Oxford, UK}
  % 11/17/15 Large deviations in random graphs
\item Northeastern U.\ workshop: Random Graphs, Simplicial Complexes, and their Appl'ns \rightloc{Boston, MA}
  % 5/21/15 Large deviations in random graphs

\item U.\ of Chicago Combinatorics and Theoretical Computer
  Science Seminar \rightloc{Chicago, IL}
  % 5/19/15 Large deviations in random graphs

\item Rutgers Discrete Math Seminar \rightloc{Piscataway, NJ}
  % 3/23/15 Large deviations in random graphs


\item ICERM workshop: Crystals, Quasicrystals and Random Networks
  \rightloc{Providence, RI}
  % 2/12/15 Large deviations in random graphs

\item[2014] Atlanta Lectures Series in Combinatorics and Graph
  Theory at Emory \rightloc{Atlanta, GA}
  % 11/1/2014 The Green-Tao theorem and a relative Szemer\'edi theorem

\item GSU Colloquium
  \rightloc{Atlanta, GA}
  % 10/31/2014 A century of progress on arithmetic progressions

\item CRM workshop: New Topics in Additive Combinatorics
  \rightloc{Montreal, QC}
  % 10/6/2014 The Green-Tao theorem and a relative Szemer\'edi theorem


\item IMA workshop: Additive and Analytic Combinatorics
  \rightloc{Minneapolis, MN}
  % 10/1/2014 The Green-Tao theorem and a relative Szemer\'edi theorem


\item Clay Math Institute workshop: Extremal and
  Probabilistic Combinatorics
  \rightloc{Oxford, UK}
  % 6/2/2014 Large deviations in random graphs: revisiting the infamous upper tail

\item Georgia Tech Combinatorics Seminar
  \rightloc{Atlanta, GA}
  % 4/4/2014 The Green-Tao theorem and a relative Szemer\'edi theorem

\item IAS Computer Science/Discrete Mathematics Seminar
  \rightloc{Princeton, NJ}
  % 3/3/2014 The Green-Tao theorem and a relative Szemer\'edi theorem

\item Oxford Combinatorial Theory Seminar \rightloc{Oxford, UK}
  % 1/2014 Sparse graph limits and scale-free networks

\item London School of Economics Discrete Mathematics and
  Game Theory Seminar \rightloc{London, UK}
  % 1/16/2014 The Green-Tao theorem and a relative Szemer\'edi theorem

\item Eurandom: Minicourse on Graph Limits
  \rightloc{Eindhoven, Netherlands} \\
  \mbox{}\qquad (6-hour minicourse co-taught with Christian Borgs)
  % 1/13/2014 Graph limits

\item Oberwolfach workshop: Combinatorics
  \rightloc{Oberwolfach, Germany}
  % 1/2014 The Green-Tao theorem and a relative Szemer\'edi theorem 20min

\item[2013] Simons Institute workshop: Neo-Classical Methods in
  Discrete Analysis \rightloc{Berkeley, CA}
  % 12/2013 The Green-Tao theorem and a relative Szemer\'edi theorem

\item Rutgers Discrete Math Seminar \rightloc{Piscataway,
    NJ}
  % 11/2013 The Green-Tao theorem and a relative Szemer\'edi theorem

\item MIT Combinatorics Seminar \rightloc{Cambridge, MA}
  % 10/16/13 The Green-Tao theorem and a relative Szemer\'edi theorem

\item Yale Combinatorics and Probability Seminar
  \rightloc{New Haven, CT}
  % 10/11/13 The Green-Tao theorem and a relative Szemer\'edi theorem

\item Microsoft Research Theory Reading Group \rightloc{Cambridge, MA}
  % 6/28/2013 Green-Tao, Szemer\'edi, and transference

\item Oberwolfach workshop: Combinatorics and Probability
  \rightloc{Oberwolfach, Germany}
  % 1/2013 large deviations in random graphs

\item[2012] MIT Combinatorics Seminar \rightloc{Cambridge, MA}   % sparse graph regularity
\item SIAM Conference on Discrete Mathematics
  \rightloc{Halifax, NS}   % sparse graph regularity
% \item[01/2011] AMS/MAA Joint Math Meetings, Undergrad Poster Session, {\em prize winner} \rightloc{New Orleans, LA}
% \item[01/2010] AMS/MAA Joint Math Meetings, Undergrad Poster Session, {\em prize winner} \rightloc{San Francisco, CA}
% \item[01/2010] AMS/MAA Joint Math Meetings Contributed Session \rightloc{San Francisco, CA}
\item[2009] MIT Combinatorics Seminar \rightloc{Cambridge, MA}
  %more sums than differences sets
\end{itemize}

%\section*{Workshop attendence (selected)}
%
%\begin{itemize}[leftmargin=.4in,itemsep=5pt,label={}]
%
%\item[2014] 
%\item Clay Math Institute: Extremal and Probabilistic Combinatorics
%\item Oberwolfach: Combinatorics \rightloc{Oberwolfach, Germany}
%
%\item[2013] Simons Institute: Neo-classical Methods in Discrete Analysis
%  \rightloc{Berkeley, CA}
%\item Banff: Geometric and Topological Graph Theory
%  \rightloc{Banff, AB}
%\item Oberwolfach: Combinatorics and Probability
%  \rightloc{Oberwolfach, Germany}
%\item UCLA/IPAM: Extremal and Probabilistic Combinatorics
%  \rightloc{Los Angeles, CA}
%\item Oberwolfach: Graph Theory \rightloc{Oberwolfach, Germany}
%\end{itemize}


%\newpage

\section*{Teaching}
\begin{longtable}[l]{rccl}
	Fa 2017 & MIT & U & 18.A34 Mathematical Problem Solving Seminar \\
	Fa 2017 & MIT & G & 18.S997 Graph Theory and Additive Combinatorics \\
	MT 2016 & Oxford & U & Geometry (tutorial) \\
	TT 2016 & Oxford & G & Polynomial Method in Combinatorics \\
	Sp 2013 & MIT & U & 18.03: Differential Equations (recitation)
\end{longtable}

{\footnotesize [U = Undergraduate, G = Graduate]}


\section*{Other Experiences and Activities}

MIT PRIMES Mentor --- 2013--2015 (Lusztig PRIMES Mentor in 2015)

Research Experience for Undergraduates at Duluth (mentor: Joe Gallian) --- Summer 2009

Deputy Leader for Canadian IMO Team --- 2008

Instructor at Canadian IMO Training Camps --- various summers and winters

Mentor at AwesomeMath Summer Program --- Summer 2007

Trainer at Math Olympiad Summer Program --- Summer 2007

Teacher at Spirit of Math Schools in Toronto --- 2005--2006

\vspace{\fill}

{\hfill \footnotesize CV updated: \today}

\end{document}